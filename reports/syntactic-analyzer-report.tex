%% abtex2-modelo-artigo.tex, v-1.9.7 laurocesar
%% Copyright 2012-2018 by abnTeX2 group at http://www.abntex.net.br/ 
%%
%% This work may be distributed and/or modified under the
%% conditions of the LaTeX Project Public License, either version 1.3
%% of this license or (at your option) any later version.
%% The latest version of this license is in
%%   http://www.latex-project.org/lppl.txt
%% and version 1.3 or later is part of all distributions of LaTeX
%% version 2005/12/01 or later.
%%
%% This work has the LPPL maintenance status `maintained'.
%% 
%% The Current Maintainer of this work is the abnTeX2 team, led
%% by Lauro César Araujo. Further information are available on 
%% http://www.abntex.net.br/
%%
%% This work consists of the files abntex2-modelo-artigo.tex and
%% abntex2-modelo-references.bib
%%

% ------------------------------------------------------------------------
% ------------------------------------------------------------------------
% abnTeX2: Modelo de Artigo Acadêmico em conformidade com
% ABNT NBR 6022:2018: Informação e documentação - Artigo em publicação 
% periódica científica - Apresentação
% ------------------------------------------------------------------------
% ------------------------------------------------------------------------

\documentclass[
	% -- opções da classe memoir --
	article,			% indica que é um artigo acadêmico
	11pt,				% tamanho da fonte
	oneside,			% para impressão apenas no recto. Oposto a twoside
	a4paper,			% tamanho do papel. 
	% -- opções da classe abntex2 --
	%chapter=TITLE,		% títulos de capítulos convertidos em letras maiúsculas
	%section=TITLE,		% títulos de seções convertidos em letras maiúsculas
	%subsection=TITLE,	% títulos de subseções convertidos em letras maiúsculas
	%subsubsection=TITLE % títulos de subsubseções convertidos em letras maiúsculas
	% -- opções do pacote babel --
	english,			% idioma adicional para hifenização
	brazil,				% o último idioma é o principal do documento
	sumario=tradicional
	]{abntex2}


% ---
% PACOTES
% ---

% ---
% Pacotes fundamentais 
% ---
\usepackage{lmodern}			% Usa a fonte Latin Modern
\usepackage[T1]{fontenc}		% Selecao de codigos de fonte.
\usepackage[utf8]{inputenc}		% Codificacao do documento (conversão automática dos acentos)
\usepackage{indentfirst}		% Indenta o primeiro parágrafo de cada seção.
\usepackage{nomencl} 			% Lista de simbolos
\usepackage{color}				% Controle das cores
\usepackage{graphicx}			% Inclusão de gráficos
\usepackage{microtype} 			% para melhorias de justificação
\usepackage{syntax}				% BNF Grammar definition
% ---
		
% ---
% Pacotes adicionais, usados apenas no âmbito do Modelo Canônico do abnteX2
% ---
\usepackage{lipsum}				% para geração de dummy text
% ---
		
% ---
% Pacotes de citações
% ---
\usepackage[brazilian,hyperpageref]{backref}	 % Paginas com as citações na bibl
\usepackage[alf]{abntex2cite}	% Citações padrão ABNT
% ---

% ---
% Configurações do pacote backref
% Usado sem a opção hyperpageref de backref
\renewcommand{\backrefpagesname}{Citado na(s) página(s):~}
% Texto padrão antes do número das páginas
\renewcommand{\backref}{}
% Define os textos da citação
\renewcommand*{\backrefalt}[4]{
	\ifcase #1 %
		Nenhuma citação no texto.%
	\or
		Citado na página #2.%
	\else
		Citado #1 vezes nas páginas #2.%
	\fi}%
% ---

% --- Informações de dados para CAPA e FOLHA DE ROSTO ---
\titulo{Implementação do Analisador Sintático}
% \tituloestrangeiro{Implementation of the Lexical Analyzer}

\autor{
Mikael Mello\thanks{Departamento de Ciência da Computação, Universidade de Brasília,
Brasília, DF, Brasil.
\mbox{\href{mailto:contact@mikaelmello.com}{contact@mikaelmello.com}} }
}

\local{Brasil}
\data{Setembro 2019}
% ---

% ---
% Configurações de aparência do PDF final

% alterando o aspecto da cor azul
\definecolor{blue}{RGB}{41,5,195}

% informações do PDF
\makeatletter
\hypersetup{
     	%pagebackref=true,
		pdftitle={\@title}, 
		pdfauthor={\@author},
    	pdfsubject={},
	   pdfcreator={Mikael Mello},
		pdfkeywords={tradutores}{relatório}{escolha do tema}, 
		colorlinks=true,       		% false: boxed links; true: colored links
    	linkcolor=blue,          	% color of internal links
    	citecolor=blue,        		% color of links to bibliography
    	filecolor=magenta,      		% color of file links
		urlcolor=blue,
		bookmarksdepth=4
}
\makeatother
% --- 

% ---
% compila o indice
% ---
\makeindex
% ---

% ---
% Altera as margens padrões
% ---
\setlrmarginsandblock{3cm}{3cm}{*}
\setulmarginsandblock{3cm}{3cm}{*}
\checkandfixthelayout
% ---

% --- 
% Espaçamentos entre linhas e parágrafos 
% --- 

% O tamanho do parágrafo é dado por:
\setlength{\parindent}{1.3cm}

% Controle do espaçamento entre um parágrafo e outro:
\setlength{\parskip}{0.2cm}  % tente também \onelineskip

% Espaçamento simples
\SingleSpacing


% ----
% Início do documento
% ----
\begin{document}

% Seleciona o idioma do documento (conforme pacotes do babel)
%\selectlanguage{english}
\selectlanguage{brazil}

% Retira espaço extra obsoleto entre as frases.
\frenchspacing 

% ----------------------------------------------------------
% ELEMENTOS PRÉ-TEXTUAIS
% ----------------------------------------------------------

%---
%
% Se desejar escrever o artigo em duas colunas, descomente a linha abaixo
% e a linha com o texto ``FIM DE ARTIGO EM DUAS COLUNAS''.
% \twocolumn[    		% INICIO DE ARTIGO EM DUAS COLUNAS
%
%---

% página de titulo principal (obrigatório)
\maketitle


% titulo em outro idioma (opcional)



% resumo em português
\begin{resumoumacoluna}
	Neste relatório é descrita a implementação do analisador sintático
	para a linguagem Afth, tendo sua análise gerada com a ferramenta
	Flex e o sintático com a ferramenta Bison.

	\vspace{\onelineskip}
	 
	\noindent
	\textbf{Palavras-chave}: tradutores. analisador sintático. flex. bison.
\end{resumoumacoluna}


% resumo em inglês
\renewcommand{\resumoname}{Abstract}
\begin{resumoumacoluna}
	\begin{otherlanguage*}{english}
		This report describes the implementation of the syntactic
		analyzer for the Afth language, generated by the Flex and Bison tools.

		\vspace{\onelineskip}
		 
		\noindent
		\textbf{Keywords}: translators. lexical analyzer. flex.
	\end{otherlanguage*}  
\end{resumoumacoluna}

% ]  				% FIM DE ARTIGO EM DUAS COLUNAS
% ---

\begin{center}\smaller
	
	%   \textbf{Data de submissão e aprovação}: 26 de agosto de 2019.
	
\end{center}

% ----------------------------------------------------------
% ELEMENTOS TEXTUAIS
% ----------------------------------------------------------
\textual

\section{Introdução}

O analisador sintático é responsável por ler uma sequência de tokens, gerada pelo analisador léxico, e gerar uma árvore de derivação a partir de uma gramática especificada. Esta árvore de derivação será então usada pelo futuro analisador semântico.

É papel do analisador sintático verificar se a entrada é parte da linguagem especificada, assim como identificar símbolos ou similares, adicionando-os a tabela de símbolos.

Como desafio, este analisador sintático foi desenvolvido em apenas um dia, e por isso um relatório detalhado de erros não foi implementado. 

Na linguagem Afth, há a implementação de conjunto como estrutura nativa. Para declarar uma variável que seja um conjunto de elementos de um tipo arbitrário, deve ser escrito \texttt{<tipo> <nome da variável>\{\};}.

Para inserir um elemento no conjunto, \texttt{<variável> << <elemento>} ou \texttt{<elemento> >> <variável>}. Para remover um elemento, a notação é \texttt{<variável> rm <elemento>}, e por fim, para verificar se um elemento pertence a um conjunto, \texttt{<elemento> in <variável>}.

\section{Funcionamento do analisador}

Este analisador é responsável por encontrar uma derivação da linguagem especificada que corresponda a sequeência de tokens lida do analisador léxico.

Para isso, o \textit{parser} gerado é do formato LR(1), o que significa que a entrada será lida da esquerda para a direita, construindo uma derivação da direita para a esquerda, sempre olhando até um símbolo seguinte da entrada de modo a identificar qual regra deve ser seguida.

Cada regra da gramática possui uma estrutura de dados correspondente, de modo que as derivações a partir desta regra são propriedades da estrutura de dados correspondente.

Assim, é criada uma árvore sintática abstrata onde o tipo de dados raiz corresponde a regra inicial da gramática, seus membros a derivação desta regra, e assim em diante.

Deste modo, a árvore pode ser facilmente atravessada por qualquer algoritmo de busca como busca em largura ou profundidade. 

Ao final do programa, esta árvore é impressa na saída padrão usando o algoritmo de busca em profundidade.

Além disso, o programa também adiciona uma entrada na tabela de símbolos toda vez que há uma declaração, que pode ser de variável ou função.

\subsection{Tratamento de erros}

Várias correções foram feitas na gramática ao longo do desenvolvimento, de modo a tanto facilitar o trabalho como corrigir conflitos.

A regra \texttt{var-declaration} agora possui uma derivação em que não há inteiro entre os colchetes. A regra \texttt{fun-declaration} agora possui uma derivação com inteiro entre os cochetes.

A regra \texttt{assignment} não é mais uma derivação direta de \texttt{statement}, mas sim de \texttt{expression}.

Havia também um conflito de shift/reduce nas regras derivadas a partir de \texttt{condition}, isto foi resolvido ao ter apenas uma regra \texttt{condition}, com variações para ambos \texttt{if} e \texttt{if else}. Além disso, no Bison é usado o modificador \texttt{\%prec} para decidir que a regra sem o \texttt{else} tem precedência, ou seja, um shift deve ser feito.

A regra \texttt{iteration} agora deriva duas regras novas, \texttt{for-iteration} e \texttt{while-iteration}, que são exatamente iguais às antigas derivações de \texttt{iteration}.

Uma regra chamada \texttt{optional-expression} foi adicionada, pois não existe a notação de uma derivação opcional em Bison, uma vez que isto não é possível em linguagens LR(1).

Houveram algumas mudanças nas ordens das regras derivadas a partir de \texttt{expression}, e além disso, \texttt{shift-expression} não possui mais uma derivação com o operador \texttt{rm}, isto foi movido para uma nova regra chamada \texttt{set-rm-expression}.

\subsection{Testes}

Para testar a identificação correta de tokens foram escritos quatro
arquivos, de extensão \texttt{.afth}, dois deles apenas com tokens
válidos e dois com alguns erros.

O primeiro, \texttt{samples/correct.lexical.afth}, contém apenas
algumas declarações de variáveis e expressões aritméticas e condicionais
simples, juntamente com uma expressão de saída \texttt{print}.
Todos os tokens são corretamente identificados, as variáveis são
interpretadas como lexemas do tipo identificador, as palavras reservadas
são identificadas de forma apropriada assim como diversos símbolos.

O segundo, \texttt{samples/correct2.lexical.afth}, possui a declaração
de uma variável do tipo conjunto de inteiros, onde todos os tokens são
devidamente agrupados em lexemas apropriados, assim como nas operações
de inserção e remoção dos elementos neste conjunto. Os símbolos de
maior e menor duplos são devidamente agrupados como um só lexema, assim
como as palavras reservadas \texttt{in} e \texttt{rm}.

O terceiro, \texttt{samples/strange-chars.lexical.afth}, possui
cadeias de caracteres não presentes na gramática da linguagem. As três
são devidamente informadas como inválidas pelo analisador léxico,
indicando corretamente a linha e a coluna onde são iniciadas.

O quarto, \texttt{samples/identifier-invalid-char.lexica.afth}, possui
testes para verificar a análise correta de identificadores. Várias letras
contínuas, separadas por símbolos como \texttt{-} ou \texttt{/}, são
devidamente interpretados como diferentes identificadores, uma vez que
o erro presente é sintático, não são reportados erros pelo analisador léxico.
Além disso, um dos identificadores possui um caractere inválido no
meio da cadeia, o analisador interpreta corretamente cada metade da cadeia
como sendo um identificador diferente e o caractere inválido é identificado
como um erro e impresso no final da execução do analisador.

\section{Gramática da linguagem}


\setlength{\grammarparsep}{0pt plus 1pt minus 1pt} % increase separation between rules
\setlength{\grammarindent}{12em} % increase separation between LHS/RHS 

\begin{grammar}
	
	<program> ::= <declaration-list>
		
	<declaration-list> ::= <declaration-list> <declaration> | <declaration>
	
	<declaration> ::= <var-declaration> | <fun-declaration>
	
	<var-declaration> ::= <type> <identifier> `;'
	\alt <type> <identifier> `[' <integer> `]' `;'
	\alt <type> <identifier> `[' `]' `;'
	\alt <type> <identifier> `\{' `\}' `;'
	
	<fun-declaration> ::= <type> <identifier> `(' <param-decs> `)' <scope>
		
	<param-decs> ::= <param-decs> `,' <param-dec> | <param-dec> | $\varepsilon$
	
	<param-dec> ::= <type> <identifier>
	\alt <type> <identifier> `[' <integer> `]'
	\alt <type> <identifier> `[' `]'
	\alt <type> <identifier> `\{' `\}'
	
	<scope> ::= `\{' <statement-list> `\}'
	
	<statement-list> ::= <statement-list> <statement> | $\varepsilon$
	
	<statement> ::= <scope>
	\alt <declaration>
	\alt <print>
	\alt <scan>
	\alt <expression> `;'
	\alt <condition>
	\alt <iteration>
	\alt <return>
	
	<print> ::= <print-type> <expression> `;'

	<scan> ::= <scan-type> <identifier> `;'

	<print-type> ::= `print' | `printc' | `printx'

	<scan-type> ::= `scan' | `scanc' | `scanf'	

	<condition> ::= `if' `(' <expression> `)' <statement>
	\alt `if' `(' <expression> `)' <statement> `else' <statement>
		
	<iteration> ::= <while-iteration> | <for-iteration>

	<while-iteration> ::= `while' `(' <expression> `)' <statement>
	
	<for-iteration> ::= `for' `(' <optional-expression> `;' <optional-expression> `;' <optional-expression> `)' <statement>

	<optional-expression> ::= <expression>
	\alt $\varepsilon$

	<return> ::= `return' <optional-expression> `;'
	
	<expression> ::= <assignment>

	<assignment> ::= <identifier> <assignment-op> <expression>
	\alt <and-expression>
	
	<and-expression> ::= <or-expression>
	\alt <and-expression> `&&' <or-expression>
	
	<or-expression> ::= <bw-and-expression>
	\alt <or-expression> `||' <bw-and-expression>
	
	<bw-and-expression> ::= <bw-or-expression>
	\alt <bw-and-expression> `&' <bw-or-expression>
	
	<bw-or-expression> ::= <bw-xor-expression>
	\alt <bw-or-expression> `|' <bw-xor-expression>
	
	<bw-xor-expression> ::= <eq-expression>
	\alt <bw-xor-expression> `^' <eq-expression>
	
	<eq-expression> ::= <rel-expression>
	\alt <eq-expression> `==' <rel-expression>
	\alt <eq-expression> `!=' <rel-expression>
	
	<rel-expression> ::= <shift-expression>
	\alt <rel-expression> <rel-op> <shift-expression>
	
	<shift-expression> ::= <set-rm-expression>
	\alt <shift-expression> `<<' <set-rm-expression>
	\alt <shift-expression> `>>' <set-rm-expression>

	<set-rm-expression> ::= <add-expression>
	\alt <set-rm-expression> `rm' <add-expression>
	
	<add-expression> ::= <mult-expression>
	\alt <add-expression> `+' <mult-expression>
	\alt <add-expression> `-' <mult-expression>
	
	<mult-expression> ::= <cast-expression>
	\alt <mult-expression> <mul-op> <cast-expression>
	
	<cast-expression> ::= <unary-expression>
	\alt `(' <type> `)' <cast-expression>
	
	<unary-expression> ::= <postfix-expression>
	\alt <unary-op> <cast-expression>
	\alt `sizeof' <cast-expression>
	
	<postfix-expression> ::= <primary-expression>
	\alt <postfix-expression> [ <expression> ]
	\alt <postfix-expression> ( <param-vals> )
		
	<param-vals> ::= <param-vals> `,' <expression> | <expression> | $\varepsilon$
	
	<primary-expression> ::= <identifier>
	\alt <constant>
	\alt <string>
	\alt `(' <expression> `)'
	
	<constant> ::= <integer>
	\alt `'' <symbol> `''
	\alt <integer> `.' <integer>
	
	<integer> ::= <digit>+
	
	<identifier> ::= <letter> \{ <letter> | <digit> | `_' \}*

	<type> ::= `void' | `bool' | `byte' | `char' | `short' | `int' | `long' | `float' | `double'
	
	<symbol> ::= any printable ascii character

	<letter> ::= `a' | `b' | ... | `z' | `A' | ... | `Z'
	
	<digit> ::= `0' | `1' | `2' | ... | `9'
	
	<assignment-op> ::= `=' | `+=' | `-=' | `*=' | `/=' | `\%='
	
	<unary-op> ::= `+' | `-' | `~' | `!'
	
	<rel-op> ::= `<' | `>' | `<=' | `>=' | `in'
	
	<mul-op> ::= `*' | `/' | `\%'
	
\end{grammar}

\section{Considerações finais}

A construção do analisador sintático se provou um bom desafio, e com mais tempo disponível para seu desenvolvimento será possível melhorá-lo e torná-lo completo, contruindo estruturas de dados mais completas para o \textit{parsing} da entrada e a possibilidade um relatório de erros detalhado.

Para a construção do analisador semântico no futuro, este analisador sintático será melhorado de modo a fornecer todas as funcionalidades necessárias para um bom tradutor.

A melhora da gramática não teria sido possível sem consultar gramáticas existentes e corretas \cite{Kernighan:1988:CPL:576122} \cite{BNFCM}. Além disso, a estrutura do analisador sintático foi inspirada em diversas soluções existentes online. \cite{youtube} \cite{Lucas1993}

\bookmarksetup{startatroot}% 

% ----------------------------------------------------------
% ELEMENTOS PÓS-TEXTUAIS
% ----------------------------------------------------------
\postextual

% ----------------------------------------------------------
% Referências bibliográficas
% ----------------------------------------------------------
\bibliography{references}

% ----------------------------------------------------------
% Glossário
% ----------------------------------------------------------
%
% Há diversas soluções prontas para glossário em LaTeX. 
% Consulte o manual do abnTeX2 para obter sugestões.
%
%\glossary

% ----------------------------------------------------------
% Apêndices
% ----------------------------------------------------------

% ---
% Inicia os apêndices
% ---
% \begin{apendicesenv}

% % ----------------------------------------------------------
% \chapter{Nullam elementum urna vel imperdiet sodales elit ipsum pharetra ligula
% ac pretium ante justo a nulla curabitur tristique arcu eu metus}
% % ----------------------------------------------------------
% \lipsum[55-56]

% \end{apendicesenv}
% ---

% ----------------------------------------------------------
% Anexos
% ----------------------------------------------------------
% \cftinserthook{toc}{AAA}
% % ---
% % Inicia os anexos
% % ---
% %\anexos
% \begin{anexosenv}

% % ---
% \chapter{Cras non urna sed feugiat cum sociis natoque penatibus et magnis dis
% parturient montes nascetur ridiculus mus}
% % ---

% \lipsum[31]

% \end{anexosenv}

% ----------------------------------------------------------
% Agradecimentos
% ----------------------------------------------------------

% \section*{Agradecimentos}
% Texto sucinto aprovado pelo periódico em que será publicado. Último 
% elemento pós-textual.

\end{document}
