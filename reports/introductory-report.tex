%% abtex2-modelo-artigo.tex, v-1.9.7 laurocesar
%% Copyright 2012-2018 by abnTeX2 group at http://www.abntex.net.br/ 
%%
%% This work may be distributed and/or modified under the
%% conditions of the LaTeX Project Public License, either version 1.3
%% of this license or (at your option) any later version.
%% The latest version of this license is in
%%   http://www.latex-project.org/lppl.txt
%% and version 1.3 or later is part of all distributions of LaTeX
%% version 2005/12/01 or later.
%%
%% This work has the LPPL maintenance status `maintained'.
%% 
%% The Current Maintainer of this work is the abnTeX2 team, led
%% by Lauro César Araujo. Further information are available on 
%% http://www.abntex.net.br/
%%
%% This work consists of the files abntex2-modelo-artigo.tex and
%% abntex2-modelo-references.bib
%%

% ------------------------------------------------------------------------
% ------------------------------------------------------------------------
% abnTeX2: Modelo de Artigo Acadêmico em conformidade com
% ABNT NBR 6022:2018: Informação e documentação - Artigo em publicação 
% periódica científica - Apresentação
% ------------------------------------------------------------------------
% ------------------------------------------------------------------------

\documentclass[
	% -- opções da classe memoir --
	article,			% indica que é um artigo acadêmico
	11pt,				% tamanho da fonte
	oneside,			% para impressão apenas no recto. Oposto a twoside
	a4paper,			% tamanho do papel. 
	% -- opções da classe abntex2 --
	%chapter=TITLE,		% títulos de capítulos convertidos em letras maiúsculas
	%section=TITLE,		% títulos de seções convertidos em letras maiúsculas
	%subsection=TITLE,	% títulos de subseções convertidos em letras maiúsculas
	%subsubsection=TITLE % títulos de subsubseções convertidos em letras maiúsculas
	% -- opções do pacote babel --
	english,			% idioma adicional para hifenização
	brazil,				% o último idioma é o principal do documento
	sumario=tradicional
	]{abntex2}


% ---
% PACOTES
% ---

% ---
% Pacotes fundamentais 
% ---
\usepackage{lmodern}			% Usa a fonte Latin Modern
\usepackage[T1]{fontenc}		% Selecao de codigos de fonte.
\usepackage[utf8]{inputenc}		% Codificacao do documento (conversão automática dos acentos)
\usepackage{indentfirst}		% Indenta o primeiro parágrafo de cada seção.
\usepackage{nomencl} 			% Lista de simbolos
\usepackage{color}				% Controle das cores
\usepackage{graphicx}			% Inclusão de gráficos
\usepackage{microtype} 			% para melhorias de justificação
\usepackage{syntax}				% BNF Grammar definition
% ---
		
% ---
% Pacotes adicionais, usados apenas no âmbito do Modelo Canônico do abnteX2
% ---
\usepackage{lipsum}				% para geração de dummy text
% ---
		
% ---
% Pacotes de citações
% ---
\usepackage[brazilian,hyperpageref]{backref}	 % Paginas com as citações na bibl
\usepackage[alf]{abntex2cite}	% Citações padrão ABNT
% ---

% ---
% Configurações do pacote backref
% Usado sem a opção hyperpageref de backref
\renewcommand{\backrefpagesname}{Citado na(s) página(s):~}
% Texto padrão antes do número das páginas
\renewcommand{\backref}{}
% Define os textos da citação
\renewcommand*{\backrefalt}[4]{
	\ifcase #1 %
		Nenhuma citação no texto.%
	\or
		Citado na página #2.%
	\else
		Citado #1 vezes nas páginas #2.%
	\fi}%
% ---

% --- Informações de dados para CAPA e FOLHA DE ROSTO ---
\titulo{Escolha do tema do projeto da disciplina Tradutores}
\tituloestrangeiro{Selection of the project theme from the Translators class}

\autor{
Mikael Mello\thanks{Departamento de Ciência da Computação, Universidade de Brasília,
Brasília, DF, Brasil.
\mbox{\href{mailto:contact@mikaelmello.com}{contact@mikaelmello.com}} }
}

\local{Brasil}
\data{Agosto 2019}
% ---

% ---
% Configurações de aparência do PDF final

% alterando o aspecto da cor azul
\definecolor{blue}{RGB}{41,5,195}

% informações do PDF
\makeatletter
\hypersetup{
     	%pagebackref=true,
		pdftitle={\@title}, 
		pdfauthor={\@author},
    	pdfsubject={},
	   pdfcreator={Mikael Mello},
		pdfkeywords={tradutores}{relatório}{escolha do tema}, 
		colorlinks=true,       		% false: boxed links; true: colored links
    	linkcolor=blue,          	% color of internal links
    	citecolor=blue,        		% color of links to bibliography
    	filecolor=magenta,      		% color of file links
		urlcolor=blue,
		bookmarksdepth=4
}
\makeatother
% --- 

% ---
% compila o indice
% ---
\makeindex
% ---

% ---
% Altera as margens padrões
% ---
\setlrmarginsandblock{3cm}{3cm}{*}
\setulmarginsandblock{3cm}{3cm}{*}
\checkandfixthelayout
% ---

% --- 
% Espaçamentos entre linhas e parágrafos 
% --- 

% O tamanho do parágrafo é dado por:
\setlength{\parindent}{1.3cm}

% Controle do espaçamento entre um parágrafo e outro:
\setlength{\parskip}{0.2cm}  % tente também \onelineskip

% Espaçamento simples
\SingleSpacing


% ----
% Início do documento
% ----
\begin{document}

% Seleciona o idioma do documento (conforme pacotes do babel)
%\selectlanguage{english}
\selectlanguage{brazil}

% Retira espaço extra obsoleto entre as frases.
\frenchspacing 

% ----------------------------------------------------------
% ELEMENTOS PRÉ-TEXTUAIS
% ----------------------------------------------------------

%---
%
% Se desejar escrever o artigo em duas colunas, descomente a linha abaixo
% e a linha com o texto ``FIM DE ARTIGO EM DUAS COLUNAS''.
% \twocolumn[    		% INICIO DE ARTIGO EM DUAS COLUNAS
%
%---

% página de titulo principal (obrigatório)
\maketitle


% titulo em outro idioma (opcional)



% resumo em português
\begin{resumoumacoluna}
	O projeto da disciplina Tradutores envolve o desenvolvimento de
	um tradutor para uma linguagem mínima que possua funcionalidades
	básicas, porém que também suporte alguma funcionalidade complexa
	não presente em C. Neste relatório é definida a estrutura da
	linguagem a ser implementada, além da motivação da escolha de
	conjuntos como a funcionalidade complexa.
	
	\vspace{\onelineskip}
	 
	\noindent
	\textbf{Palavras-chave}: tradutores. relatório. 
	escolha do tema. linguagens. conjuntos.
\end{resumoumacoluna}


% resumo em inglês
\renewcommand{\resumoname}{Abstract}
\begin{resumoumacoluna}
	\begin{otherlanguage*}{english}
		The project of the Translators class involves developing a
		translator for a minimal language that supports basic operations,
		but that also implements a complex feature that is not present
		in C. The structure of the language to be implemented along with
		the motivation behind choosing sets as the complex feature is
		defined in this report.	
		\vspace{\onelineskip}
		 
		\noindent
		\textbf{Keywords}: translators. report. theme selection.
		languages. sets.
	\end{otherlanguage*}  
\end{resumoumacoluna}

% ]  				% FIM DE ARTIGO EM DUAS COLUNAS
% ---

\begin{center}\smaller
	
	%   \textbf{Data de submissão e aprovação}: 26 de agosto de 2019.
	
\end{center}

% ----------------------------------------------------------
% ELEMENTOS TEXTUAIS
% ----------------------------------------------------------
\textual

% ----------------------------------------------------------
% Introdução
% ----------------------------------------------------------
\section{Introdução}

O projeto prático da disciplina Tradutores tem como principais objetivos o entendimento
dos aspectos teóricos e práticos da implementação de tradutores. Assim, a principal
para o aluno é a implementação de um tradutor de uma linguagem simples, similar à C,
porém com alguma funcionalidade complexa que não esteja presente em C,
a ser definida pelo aluno.

A linguagem simplificada deve, obrigatoriamente, possuir as estruturas básicas de uma
linguagem de programação: comandos de leitura e escrita, comando condicional,
comando de repetição, tratamento de expressões aritméticas e booleanas e chamada a
subrotinas.

Estas, e mais algumas, serão implementadas, juntamente com o suporte embutido de conjunto
como o tipo de uma variável. Um conjunto é uma coleção de elementos distinguíveis, chamados
elementos, não pode possuir o mesmo objeto mais de uma vez e seus elementos não são ordenados
\cite[Apêndice B.1]{Cormen:2009:IAT:1614191}. Neste artigo, e em outros que detalhem a implementação da linguagem e do tradutor, a linguagem
será chamada de Afth, originado de \textit{August 5th}.

% ----------------------------------------------------------
% Seção de explicações
% ----------------------------------------------------------
\section{Motivação para a escolha}

Conjuntos são amplamente usados no campo da Ciência da Computação
como um todo. Se trata de um conceito matemático extremamente importante
com várias aplicações em computação.

Um banco de dados relacional pode ser descrito como estruturar um banco de dados
como uma relação entre conjuntos, onde operações como união, interseção, diferença,
junção e várias outras são fundamentais, todas estudadas na área da Álgebra Relacional
e aproveitadas na implementação de bancos de dados.

Na área linguagens formais, uma linguagem é um conjunto de cadeias, assim como o
seu alfabeto é um conjunto de símbolos, dentre várias outras aplicações. Esta área
da Ciência da Computação é importante tanto para aspectos teóricos, como decidibilidade,
computabilidade, complexidade, entre outros conceitos, como em aplicações, tendo exemplos
de processamento de linguagens, reconhecimento de padrões, entre outros.

Por estas e outras aplicações, uma implementação embutida de conjuntos em uma linguagem
é de interesse do programador, que não precisa se preocupar em implementar o código
que realize estas operações de forma eficiente, uma vez que a linguagem possui
suporte nativo e seu tradutor tratará de otimizá-las.

\section{Descrição da linguagem}

Afth é uma linguagem de programação compilada de propósito geral, estruturada, imperativa,
procedural e possui tipagem fraca e estática.

Os tipos primitivos de uma variável são:

\begin{itemize}
	\item \texttt{byte}, \texttt{bool}, \texttt{char}: Inteiros sinalizados em
	      complemento de 2, possuindo 8 bits de tamanho
	\item \texttt{short}, \texttt{int}, \texttt{long}: Inteiros sinalizados em
	      complemento de 2, possuindo 16, 32 e 64 bits de tamanho, respectivamente.
	\item \texttt{float}, \texttt{double}: Números reais no formato de ponto
	      flutuante IEEE 754 de precisão simples e dupla, respectivamente.
\end{itemize}

Um programa Afth é composto por declarações de variáveis ou funções, sendo necessário
que haja a declaração de uma função chamada \texttt{main} que retorne uma variável do tipo
\texttt{int}. Dentro de funções, estão conjuntos de expressões que podem ser de repetição,
condicionais, atribuições, aritméticas, lógicas, leitura e escrita, retorno de função,
declarações de variáveis, entre outras definidas na gramática.

Cada variável possui um tipo primitivo, se em sua declaração houver um par de chaves
\texttt{\{\}} logo após seu identificador, isto significa que a variável é um conjunto
de elementos do tipo primitivo definido, se possuir um par de colchetes \texttt{[]} 
com um número dentro, é um vetor de tamanho fixo cujos elementos são do tipo definido.

A linguagem implementa comandos condicionais \texttt{if-else if-else}, laços de repetição
no formato \texttt{for} e \texttt{while}, tratamento de expressões aritméticas com operadores
\texttt{+}, \texttt{-}, \texttt{*}, \texttt{/}, \texttt{\%}, operadores bit a bit \texttt{\&},
\texttt{|}, \texttt{\^}, \texttt{\~}, \texttt{<<}, \texttt{>>}, operadores lógicos \texttt{||},
\texttt{\&\&}, \texttt{!}, operadores relacionais \texttt{==}, \texttt{>}, \texttt{<}, \texttt{!=},
\texttt{>=}, \texttt{<=}, operadores de atribuição \texttt{=}, \texttt{+=}, \texttt{-=}, \texttt{*=},
\texttt{/=}, \texttt{\%=}, comandos de escrita \texttt{print}, \texttt{printc}, \texttt{printx} e
de leitura \texttt{scan}, \texttt{scanc} e \texttt{scanf}.

Para verificar se um elemento \texttt{x} está no conjunto \texttt{A}, há o operador \texttt{in}
(\texttt{x in A}), para inserir um elemento no conjunto, \texttt{x >> A} ou \texttt{A << x}. Para
remover, \texttt{A rm x}. Além disso, serão implementadas as operações de união
(\texttt{A | B}), interseção (\texttt{A \& B}) e diferença de conjuntos (\texttt{A - B}).

\subsection{Gramática}

A gramática da linguagem Afth é inspirada na gramática de C
\cite[Seção A13]{Kernighan:1988:CPL:576122} e C-Minus \cite{BNFCM}.


\setlength{\grammarparsep}{0pt plus 1pt minus 1pt} % increase separation between rules
\setlength{\grammarindent}{12em} % increase separation between LHS/RHS 

\begin{grammar}
	
	<program> ::= <declaration-list>
		
	<declaration-list> ::= <declaration> `;' <declaration-list> | <declaration>
	
	<declaration> ::= <var-declaration> | <fun-declaration>
	
	<var-declaration> ::= <type> <identifier> `;'
	\alt <type> <identifier> `[' <integer> `]' `;'
	\alt <type> <identifier> `\{' `\}' `;'
	
	<fun-declaration> ::= <type> <identifier> `(' <params> `)' <scope>
		
	<params> ::= <params> `,' <param> | <param> | $\varepsilon$
	
	<param> ::= <type> <identifier>
	\alt <type> <identifier> `[' `]'
	\alt <type> <identifier> `\{' `\}'
	
	<scope> ::= `\{' <statement-list> `\}'
	
	<statement-list> ::= <statement-list> <statement> | $\varepsilon$
	
	<statement> ::= <scope> | <expression> | <condition> | <iteration> | <return>
	
	<condition> ::= `if' `(' <expression> `)' <statement> <condition-mid>
		
	<condition-mid> ::= `else if `(' <expression> `)' <statement> <condition-mid>
	\alt <condition-end>
	
	<condition-end> ::= `else' <statement> | $\varepsilon$
	
	<iteration> ::= `while' `(' <expression> `)' <statement>
	\alt `for' `(' <expression>? `;' <expression>? `;' <expression>? `)' <statement>
	
	<return> ::= `return' <expression>? `;'
	
	<assignment> ::= <identifier> <assignment-op> <expression> `;'
	
	<expression> ::= <var-declaration> | <assignment> | <and-expression>
	
	<and-expression> ::= <or-expression>
	\alt <and-expression> `&&' <or-expression>
	
	<or-expression> ::= <bw-or-expression>
	\alt <or-expression> `||' <bw-or-expression>
	
	<bw-or-expression> ::= <bw-xor-expression>
	\alt <bw-or-expression> `|' <bw-xor-expression>
	
	<bw-xor-expression> ::= <bw-and-expression>
	\alt <bw-xor-expression> `^' <bw-and-expression>
	
	<bw-and-expression> ::= <eq-expression>
	\alt <bw-and-expression> `&' <eq-expression>
	
	<eq-expression> ::= <relational-expression>
	\alt <eq-expression> `==' <rel-expression>
	\alt <eq-expression> `!=' <rel-expression>
	
	<rel-expression> ::= <shift-expression>
	\alt <rel-expression> <rel-op> <shift-expression>
	
	<shift-expression> ::= <add-expression>
	\alt <shift-expression> `<<' <add-expression>
	\alt <shift-expression> `>>' <add-expression>
	\alt <shift-expression> `rm' <add-expression>
	
	<add-expression> ::= <mult-expression>
	\alt <add-expression> `+' <mult-expression>
	\alt <add-expression> `-' <mult-expression>
	
	<mult-expression> ::= <cast-expression>
	\alt <mult-expression> <mul-op> <cast-expression>
	
	<cast-expression> ::= <unary-expression>
	\alt `(' <type> `)' <cast-expression>
	
	<unary-expression> ::= <postfix-expression>
	\alt <unary-op> <cast-expression>
	\alt sizeof <cast-expression>
	
	<postfix-expression> ::= <primary-expression>
	\alt <postfix-expression> [ <expression> ]
	\alt <postfix-expression> ( <param-values> )
		
	<param-values> ::= <param-values> `,' <expression> | <expression> | $\varepsilon$
	
	<primary-expression> ::= <identifier>
	\alt <constant>
	\alt <string>
	\alt `(' <expression> `)'
	
	<constant> ::= <integer>
	\alt `'' <symbol> `''
	\alt <integer> `.' <integer>
	
	<integer> ::= <digit>+
	
	<identifier> ::= <letter> \{ <letter> | <digit> | `_' \}*
	t
	<type> ::= `bool' | `byte' | `char' | `short' | `int' | `long' | `float' | `double'
	
	<symbol> ::= any printable ascii character

	<letter> ::= `a' | `b' | ... | `z' | `A' | ... | `Z'
	
	<digit> ::= `0' | `1' | `2' | ... | `9'
	
	<assignment-op> ::= `=' | `+=' | `-=' | `*=' | `/=' | `\%='
	
	<unary-op> ::= `+' | `-' | `~' | `!'
	
	<rel-op> ::= `<' | `>' | `<=' | `>=' | `in'
	
	<mul-op> ::= `*' | `/' | `\%'
	
\end{grammar}

% ---
% Finaliza a parte no bookmark do PDF, para que se inicie o bookmark na raiz
% ---
\bookmarksetup{startatroot}% 
% ---

% ---
% Conclusão
% ---
\section{Considerações finais}

Definir a linguagem-alvo do tradutor a ser implementado é fundamental para o 
desenvolvimento do projeto, pois nesta fase de planejamento são observadas de antemão
possíveis dificuldades futuras durante a implementação.

Foi especificada a gramática de Afth, implementando estruturas básicas de uma
linguagem de programação procedural, juntamente com uma sintaxe que permite
a implementação da funcionalidade de conjuntos como um tipo de dados embutido.

É possível observar que a implementação de conjuntos trará dificuldades
ao conciliar versatilidade e implementação de funcionalidades juntamente
com eficiência, de modo que a escolha do uso de conjuntos em algum programa
não o tornará mais lento.

% ----------------------------------------------------------
% ELEMENTOS PÓS-TEXTUAIS
% ----------------------------------------------------------
\postextual

% ----------------------------------------------------------
% Referências bibliográficas
% ----------------------------------------------------------
\bibliography{references}

% ----------------------------------------------------------
% Glossário
% ----------------------------------------------------------
%
% Há diversas soluções prontas para glossário em LaTeX. 
% Consulte o manual do abnTeX2 para obter sugestões.
%
%\glossary

% ----------------------------------------------------------
% Apêndices
% ----------------------------------------------------------

% ---
% Inicia os apêndices
% ---
% \begin{apendicesenv}

% % ----------------------------------------------------------
% \chapter{Nullam elementum urna vel imperdiet sodales elit ipsum pharetra ligula
% ac pretium ante justo a nulla curabitur tristique arcu eu metus}
% % ----------------------------------------------------------
% \lipsum[55-56]

% \end{apendicesenv}
% ---

% ----------------------------------------------------------
% Anexos
% ----------------------------------------------------------
% \cftinserthook{toc}{AAA}
% % ---
% % Inicia os anexos
% % ---
% %\anexos
% \begin{anexosenv}

% % ---
% \chapter{Cras non urna sed feugiat cum sociis natoque penatibus et magnis dis
% parturient montes nascetur ridiculus mus}
% % ---

% \lipsum[31]

% \end{anexosenv}

% ----------------------------------------------------------
% Agradecimentos
% ----------------------------------------------------------

% \section*{Agradecimentos}
% Texto sucinto aprovado pelo periódico em que será publicado. Último 
% elemento pós-textual.

\end{document}
