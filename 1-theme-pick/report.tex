%% abtex2-modelo-artigo.tex, v-1.9.7 laurocesar
%% Copyright 2012-2018 by abnTeX2 group at http://www.abntex.net.br/ 
%%
%% This work may be distributed and/or modified under the
%% conditions of the LaTeX Project Public License, either version 1.3
%% of this license or (at your option) any later version.
%% The latest version of this license is in
%%   http://www.latex-project.org/lppl.txt
%% and version 1.3 or later is part of all distributions of LaTeX
%% version 2005/12/01 or later.
%%
%% This work has the LPPL maintenance status `maintained'.
%% 
%% The Current Maintainer of this work is the abnTeX2 team, led
%% by Lauro César Araujo. Further information are available on 
%% http://www.abntex.net.br/
%%
%% This work consists of the files abntex2-modelo-artigo.tex and
%% abntex2-modelo-references.bib
%%

% ------------------------------------------------------------------------
% ------------------------------------------------------------------------
% abnTeX2: Modelo de Artigo Acadêmico em conformidade com
% ABNT NBR 6022:2018: Informação e documentação - Artigo em publicação 
% periódica científica - Apresentação
% ------------------------------------------------------------------------
% ------------------------------------------------------------------------

\documentclass[
	% -- opções da classe memoir --
	article,			% indica que é um artigo acadêmico
	11pt,				% tamanho da fonte
	oneside,			% para impressão apenas no recto. Oposto a twoside
	a4paper,			% tamanho do papel. 
	% -- opções da classe abntex2 --
	%chapter=TITLE,		% títulos de capítulos convertidos em letras maiúsculas
	%section=TITLE,		% títulos de seções convertidos em letras maiúsculas
	%subsection=TITLE,	% títulos de subseções convertidos em letras maiúsculas
	%subsubsection=TITLE % títulos de subsubseções convertidos em letras maiúsculas
	% -- opções do pacote babel --
	english,			% idioma adicional para hifenização
	brazil,				% o último idioma é o principal do documento
	sumario=tradicional
	]{abntex2}


% ---
% PACOTES
% ---

% ---
% Pacotes fundamentais 
% ---
\usepackage{lmodern}			% Usa a fonte Latin Modern
\usepackage[T1]{fontenc}		% Selecao de codigos de fonte.
\usepackage[utf8]{inputenc}		% Codificacao do documento (conversão automática dos acentos)
\usepackage{indentfirst}		% Indenta o primeiro parágrafo de cada seção.
\usepackage{nomencl} 			% Lista de simbolos
\usepackage{color}				% Controle das cores
\usepackage{graphicx}			% Inclusão de gráficos
\usepackage{microtype} 			% para melhorias de justificação
% ---
		
% ---
% Pacotes adicionais, usados apenas no âmbito do Modelo Canônico do abnteX2
% ---
\usepackage{lipsum}				% para geração de dummy text
% ---
		
% ---
% Pacotes de citações
% ---
\usepackage[brazilian,hyperpageref]{backref}	 % Paginas com as citações na bibl
\usepackage[alf]{abntex2cite}	% Citações padrão ABNT
% ---

% ---
% Configurações do pacote backref
% Usado sem a opção hyperpageref de backref
\renewcommand{\backrefpagesname}{Citado na(s) página(s):~}
% Texto padrão antes do número das páginas
\renewcommand{\backref}{}
% Define os textos da citação
\renewcommand*{\backrefalt}[4]{
	\ifcase #1 %
		Nenhuma citação no texto.%
	\or
		Citado na página #2.%
	\else
		Citado #1 vezes nas páginas #2.%
	\fi}%
% ---

% --- Informações de dados para CAPA e FOLHA DE ROSTO ---
\titulo{Escolha do tema do projeto da disciplina Tradutores}
\tituloestrangeiro{Selection of the project theme from the Translators class}

\autor{
Mikael Mello\thanks{Departamento de Ciência da Computação, Universidade de Brasília,
Brasília, DF, Brasil.
\mbox{\href{mailto:contact@mikaelmello.com}{contact@mikaelmello.com}} }
}

\local{Brasil}
\data{Agosto 2019}
% ---

% ---
% Configurações de aparência do PDF final

% alterando o aspecto da cor azul
\definecolor{blue}{RGB}{41,5,195}

% informações do PDF
\makeatletter
\hypersetup{
     	%pagebackref=true,
		pdftitle={\@title}, 
		pdfauthor={\@author},
    	pdfsubject={},
	   pdfcreator={Mikael Mello},
		pdfkeywords={tradutores}{relatório}{escolha do tema}, 
		colorlinks=true,       		% false: boxed links; true: colored links
    	linkcolor=blue,          	% color of internal links
    	citecolor=blue,        		% color of links to bibliography
    	filecolor=magenta,      		% color of file links
		urlcolor=blue,
		bookmarksdepth=4
}
\makeatother
% --- 

% ---
% compila o indice
% ---
\makeindex
% ---

% ---
% Altera as margens padrões
% ---
\setlrmarginsandblock{3cm}{3cm}{*}
\setulmarginsandblock{3cm}{3cm}{*}
\checkandfixthelayout
% ---

% --- 
% Espaçamentos entre linhas e parágrafos 
% --- 

% O tamanho do parágrafo é dado por:
\setlength{\parindent}{1.3cm}

% Controle do espaçamento entre um parágrafo e outro:
\setlength{\parskip}{0.2cm}  % tente também \onelineskip

% Espaçamento simples
\SingleSpacing


% ----
% Início do documento
% ----
\begin{document}

% Seleciona o idioma do documento (conforme pacotes do babel)
%\selectlanguage{english}
\selectlanguage{brazil}

% Retira espaço extra obsoleto entre as frases.
\frenchspacing 

% ----------------------------------------------------------
% ELEMENTOS PRÉ-TEXTUAIS
% ----------------------------------------------------------

%---
%
% Se desejar escrever o artigo em duas colunas, descomente a linha abaixo
% e a linha com o texto ``FIM DE ARTIGO EM DUAS COLUNAS''.
% \twocolumn[    		% INICIO DE ARTIGO EM DUAS COLUNAS
%
%---

% página de titulo principal (obrigatório)
\maketitle


% titulo em outro idioma (opcional)



% resumo em português
\begin{resumoumacoluna}
   O projeto da disciplina Tradutores envolve o desenvolvimento de
   um tradutor para uma linguagem mínima que possua funcionalidades
   básicas, porém que também suporte alguma funcionalidade complexa
   não presente em C. Neste relatório é definida a estrutura da
   linguagem a ser implementada, além da motivação da escolha de
   conjuntos como a funcionalidade complexa.

 \vspace{\onelineskip}
 
 \noindent
 \textbf{Palavras-chave}: tradutores. relatório. 
 						  escolha do tema. linguagens. conjuntos.
\end{resumoumacoluna}


% resumo em inglês
\renewcommand{\resumoname}{Abstract}
\begin{resumoumacoluna}
 \begin{otherlanguage*}{english}
   The project of the Translators class involves developing a
   translator for a minimal language that supports basic operations,
   but that also implements a complex feature that is not present
   in C. The structure of the language to be implemented along with
   the motivation behind choosing sets as the complex feature is
   defined in this report.	
   \vspace{\onelineskip}
 
   \noindent
   \textbf{Keywords}: translators. report. theme selection.
   					  languages. sets.
 \end{otherlanguage*}  
\end{resumoumacoluna}

% ]  				% FIM DE ARTIGO EM DUAS COLUNAS
% ---

\begin{center}\smaller

%   \textbf{Data de submissão e aprovação}: 26 de agosto de 2019.

\end{center}

% ----------------------------------------------------------
% ELEMENTOS TEXTUAIS
% ----------------------------------------------------------
\textual

% ----------------------------------------------------------
% Introdução
% ----------------------------------------------------------
\section{Introdução}

Ao considerar as diversas possibilidades acerca da linguagem para qual será
implementado o tradutor, algumas das principais questões levantadas são:

\begin{itemize}
	\item Qual o nível de dificuldade esperado para concluir os objetivos definidos?
	\item O resultado final do projeto possui potencial para ser relevante no futuro?
\end{itemize}

Deste modo, se decide que a linguagem escolhida será uma que possui casos de uso reais
e relevantes na comunidade, tenha sua especificação precisamente definida de modo
a evitar ambiguidades e acelerar o desenvolvimento do projeto, e que um subconjunto
de suas funcionalidades ainda possua utilidade significativa.

Por isso, evita-se a escolha de linguagens inventadas, com especificações ambíguas e
 grande potencial de má interpretação ou que não seja factível que o projeto final,
 dentro das limitações de tempo impostas pelo semestre letivo, tenha implementado
 funcionalidades ditas essenciais para a linguagem.

Assim, Haskell, em sua revisão 98, é escolhida. Uma linguagem puramente funcional,
criada com objetivos que incluem: ser adequada a, além do desenvolvimento de aplicações,
o ensino e a pesquisa; ser completamente descrita por uma publicação de uma sintaxe
formal e sua semântica; ser disponível livremente para todos; reduzir diversidade
desnecessária entre linguagens de programação funcionais. \cite[Prefácio]{Haskell98}.

% ----------------------------------------------------------
% Seção de explicações
% ----------------------------------------------------------
\section{Descrição da linguagem}

De forma abstrata, pode-se descrever a estrutura sintática e semântica
de Haskell da seguinte forma:

No nível mais externo, um programa Haskell é um conjunto de
\textit{módulos}. Módulos fornecem um modo de controlar \textit{namespaces}
e reutilizar \textit{software} em grandes programas.

Um módulo é composto por uma coleção de \textit{declarações}, que podem
ser de vários tipos. Declarações definem coisas como valores comuns,
tipos de dados, tipos de classes e informações sobre invariabilidade.

Logo após, existem \textit{expressões}. Expressões denotam um \textit{valor}
e possuem um \textit{tipo estático}, elas são o coração de um programa.

No mais baixo nível, está a estrutura léxica de Haskell. Ela captura a 
representação concreta de programas Haskell em arquivos de texto.

A notação usada para apresentar sua estrutura léxica é:


\begin{center}
\begin{tabular}{ c l } 
	[ $ pattern $ ]     & opcional \\ 
	\{ $ pattern $ \} & zero ou mais repetições \\ 
	( $ pattern $ )     & agrupamento \\ 
	$pat_1$ | $pat_2$  & escolha \\ 
	$ pat_{<pat'>} $   & diferença - elementos gerados por $pat$ \\
					   & exceto aqueles gerados por $pat'$ \\ 
	\texttt{fibonacci} & sintaxe terminal em fonte monoespaçada \\ 
\end{tabular}
\end{center}
	
Assim, a gramática, modificada para apenas aceietar caracteres
ASCII, é definida pelas regras:

\begin{tabular}{ l l l } 
	$ program $   & $ \rightarrow $ & \{ $lexeme$ | $whitespace$ \} \\
	$ lexeme $    & $ \rightarrow $ & $qvarid$ | $qconid$ | $qvarsym$ | $qconsym$ \\
				  & |				  & $literal$ | $special$ | $servedop$ | $reservedid$ \\
	$ literal $   & $ \rightarrow $ & $integer$ | $float$ | $char$ | $string$ \\
	$ special $   & $ \rightarrow $ & ( | ) | , | ; | [ | ] | ` | \{ | \} \\
	$ comment $   & $ \rightarrow $ & $dashes$ [ $any_{<symbol>}$ \{ $any$ \}] $newline$ \\
	$ dashes $    & $ \rightarrow $ & - - \{ - \} \\
	$ opencom $   & $ \rightarrow $ & \{ - \\
	$ closecom $  & $ \rightarrow $ & - \} \\
	$ ncomment $  & $ \rightarrow $ & $opencom$ $ANYseq$ \{ $ncomment$ $ANYseq$ \} $closecom$ \\
	$ ANYseq$     & $ \rightarrow $ & $\{ANY\}_{<\{ANY\} ( opencom | closecom ) \{ANY\}>}$ \\
	$ ANY $       & $ \rightarrow $ & $graphic$ | $whitechar$ \\
	$ any $       & $ \rightarrow $ & $graphic$ | $space$ | $tab$ \\
	$ graphic $   & $ \rightarrow $ & $small$ | $large$ | $symbol$ | $digit$ | $special$ | : | " | ' \\
	$ small $	  & $ \rightarrow $ & $ascSmall$ | $\_$ \\
	$ ascSmall $  & $ \rightarrow $ & \texttt{a} | \texttt{b} | \texttt{...} | \texttt{z} \\
	$ large $     & $ \rightarrow $ & $ascLarge$ \\
	$ ascLarge $  & $ \rightarrow $ & \texttt{A} | \texttt{B} | \texttt{...} | \texttt{Z} \\
	$ symbol $    & $ \rightarrow $ & $ascSymbol$ \\
	$ ascSymbol $ & $ \rightarrow $ & ! | \# | \$ | \% | \& | * | + | . | / | < | = | > | ? | @ \\
				  & |				& \textbackslash | \^ | \\
\end{tabular}


\section{Motivação para a escolha}


% ---
% Finaliza a parte no bookmark do PDF, para que se inicie o bookmark na raiz
% ---
\bookmarksetup{startatroot}% 
% ---

% ---
% Conclusão
% ---
\section{Considerações finais}

% ----------------------------------------------------------
% ELEMENTOS PÓS-TEXTUAIS
% ----------------------------------------------------------
\postextual

% ----------------------------------------------------------
% Referências bibliográficas
% ----------------------------------------------------------
\bibliography{references}

% ----------------------------------------------------------
% Glossário
% ----------------------------------------------------------
%
% Há diversas soluções prontas para glossário em LaTeX. 
% Consulte o manual do abnTeX2 para obter sugestões.
%
%\glossary

% ----------------------------------------------------------
% Apêndices
% ----------------------------------------------------------

% ---
% Inicia os apêndices
% ---
% \begin{apendicesenv}

% % ----------------------------------------------------------
% \chapter{Nullam elementum urna vel imperdiet sodales elit ipsum pharetra ligula
% ac pretium ante justo a nulla curabitur tristique arcu eu metus}
% % ----------------------------------------------------------
% \lipsum[55-56]

% \end{apendicesenv}
% ---

% ----------------------------------------------------------
% Anexos
% ----------------------------------------------------------
% \cftinserthook{toc}{AAA}
% % ---
% % Inicia os anexos
% % ---
% %\anexos
% \begin{anexosenv}

% % ---
% \chapter{Cras non urna sed feugiat cum sociis natoque penatibus et magnis dis
% parturient montes nascetur ridiculus mus}
% % ---

% \lipsum[31]

% \end{anexosenv}

% ----------------------------------------------------------
% Agradecimentos
% ----------------------------------------------------------

% \section*{Agradecimentos}
% Texto sucinto aprovado pelo periódico em que será publicado. Último 
% elemento pós-textual.

\end{document}
